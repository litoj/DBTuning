\documentclass[11pt]{scrartcl}

\usepackage[top=1.5cm]{geometry}
\usepackage{url}
\usepackage{float}
\usepackage{listings}
\usepackage{xcolor}

\setlength{\parindent}{0em}
\setlength{\parskip}{0.5em}

\newcommand{\youranswerhere}{[Your answer goes here \ldots]}
\renewcommand{\thesubsection}{\arabic{subsection}}

\lstdefinestyle{dbtsql}{
  language=SQL,
  basicstyle=\small\ttfamily,
  keywordstyle=\color{magenta!75!black},
  stringstyle=\color{green!50!black},
  showspaces=false,
  showstringspaces=false,
  commentstyle=\color{gray}}

\title{
  \textbf{\large Assignment 1} \\
  Uploading Data to the Database \\
  {\large Database Tuning}}

\author{
  Group Name (e.g. A1, B5, B3) \\
  \large Lastname1 Firstname1, StudentID1 \\
  \large Lastname2 Firstname2, StudentID2 \\
  \large Lastname3 Firstname3, StudentID3
}

\begin{document}

\maketitle

\subsection*{Experimental Setup}

Describe your experimental setup in a few lines.

\youranswerhere{}

\subsection*{Straightforward Implementation}

\paragraph{Implementation}

Describe in a few lines how this approach works and show the query that you use. You may also show small (!) code snippets if you think they help the understanding.

\youranswerhere{}

\begin{lstlisting}[style=dbtsql]
[Your SQL query goes here ...]
\end{lstlisting}

\subsection*{Efficient Approaches}

\subsubsection*{Efficient Approach 1: (NAME)}

\paragraph{Implementation}

Describe in a few lines how this approach works and show the query that you use. You may also show small (!) code snippets if you think they help the understanding.

\youranswerhere{}

\begin{lstlisting}[style=dbtsql]
[Your SQL query goes here ...]
\end{lstlisting}

\paragraph{Why is this approach efficient?}

Explain, why this approach is more efficient than the straightforward approach. Where does the system save the time? Be clear and precise!

\emph{Important:} Cite the references that you used to answer this question, for example, using footnotes or the References section at the end of the report.

\youranswerhere{}

\paragraph{Tuning principle}

Which tuning principle did you apply? Pick the one that describes this approach best (``think globally, fix locally'' is too general).

\youranswerhere{}

\subsubsection*{Portability}

\paragraph{Implementation}

Describe in a few lines what you needed to change (esp. towards SQL). You may also show small (!) code snippets if you think they help the understanding.

\youranswerhere

\begin{lstlisting}[style=dbtsql]
[Your SQL query goes here ...]
\end{lstlisting}

\paragraph{Did you observe performance differences. If so: Why. If not: Why not?}



\youranswerhere{}

\subsection*{Runtime Experiment}

\paragraph{Notes}

\begin{itemize}
  \item For the straightforward approach you are allowed to import only a subset of the tuples (e.g., 10.000 tuples) and estimate the overall runtime. The timings for all other approaches should be real measurements over the whole data set.
  \item Specify the setting of the experiment, i.e., where is the database server (local machine, database server at the department), where is the client (wired/wireless network of the department)?
\end{itemize}

\begin{table}[H]
  \centering
  \begin{tabular}{l|r}
    Approach          & Runtime [sec] \tabularnewline
    \hline
    Straightforward   & \ldots \tabularnewline
    Approach (NAME) & \ldots \tabularnewline
    Local Approach & \ldots \tabularnewline
  \end{tabular}
\end{table}

\subsection*{Time Spent on this Assignment}

Time in hours per person: \textbf{XXX}

\subsection*{References}

\begin{table}[H]
  \centering
  \begin{tabular}{c}
    \hline
    \textbf{Important:} Reference your information sources! \tabularnewline
    Remove this section if you use footnotes to reference your information sources. \tabularnewline
    \hline
  \end{tabular}
\end{table}

\end{document}
